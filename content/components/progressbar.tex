%!TEX root = ../../main.tex

\chapter{Progress Bar}
\index{Progress Bar}
\index{Bars!Progress Bar}
\label{cha:progress_bar}

The Android framework includes a widget called \androidinline{ProgressBar} which can be used both as an activity indicator (see Figure \ref{fig:dialog_waiting}), and as an actual progress bar. Both usages of the word will henceforth be referred to as just \androidinline{ProgressBar}, unless otherwise made explicit. Both are great at indicating that the application is not frozen and that something is going on. The ProgressBar (as a horizontal progress bar) should generally be used when there is a reliable way of calculating the actual progress on the running task a seen in \figref{fig:horizontal_progressbar}. The Activity indicator should generally be used when there is no way of telling how long there will be until the task is complete as seen in \figref{fig:spinning_progressbar}.

\begin{figure}[!htbp]
    \centering
    \begin{subfigure}[t]{0.4\textwidth}
        \centering
        \includegraphics[scale=0.5]{progressbar}
    \caption{A spinning progressbar}
    \label{fig:spinning_progressbar}
    \end{subfigure}
    \hspace{5em} 
    \begin{subfigure}[t]{0.4\textwidth}
        \centering
        \includegraphics[scale=0.5]{progressbar_horizontal}
    \caption{A horizontal progressbar}
    \label{fig:horizontal_progressbar}
    \end{subfigure}
    
    \caption{Examples of progressbars}
    \label{fig:progressbars}
\end{figure}
