%!TEX root = ../../main.tex

\chapter{Icons}
\index{Icons}

Throughout the \giraf-software suite different icons will be used to reference different applications and functionalities. Refer to the sections below to see how and when to use these icons.

\section{\giraf Software Suite Icon}
\index{Icons}
The different applications in the \giraf software suite may want to refer to the software suite itself. To do this, the overall application icon seen in \figref{fig:overall_application_icon} can be used. This icon will also be used in some situations for indicating activity. \todo{Refer to a section describing activity indicators}

\begin{figure}[h]
	\centering
	\includegraphics[scale=0.25]{placeholder}
	\caption{Overall application icon for the \giraf software suite}
	\label{fig:overall_application_icon}
\end{figure}

\section{Application-specific Icons}
\index{Icons}
Each application in the \giraf software suite must have its own icon. This icon should reflect the content and functionality of the application and must not be ambiguous. All application icons should furthermore use the icon-base seen in \figref{fig:application_specific_icon_base}. 
\\\\
Rendered icons must be available in sizes defined in the \href{http://developer.android.com/design/style/iconography.html}{Android Iconography article}. The foreground of the icon must be clear in any of these sizes. Furthermore, the icon must not appear smudged or otherwise distorted on any scale.

\begin{figure}[h]
	\centering
	\includegraphics[width=0.15\textwidth]{application_specific_icon_base}
	\caption{Base for all application specific icon}
	\label{fig:application_specific_icon_base}
\end{figure}

\section{Icons used to represent functionality}
\index{Icons}
Specific icons may be used in buttons (See \texttt{GirafButton} in \gc) to represent functionality. The foreground of these icons must be gray-scaled and be quadratic (square). The functionality that the icon represents should be reflected in the icons itself. The icons should use the icon-base seen in \figref{fig:functionality_specific_icon_base}.

\begin{note}
	Please be aware that the actual icons should not be designed/rendered using the base as described above. Instead use the \texttt{GirafButton} from the \gc library. This will allow you to only design the actual foreground and not worry about the background (base).
\end{note}

\begin{figure}[h]
	\centering
	\includegraphics[width=0.15\textwidth]{functionality_specific_icon_base}
	\caption{Base for all functionality-specific icons}
	\label{fig:functionality_specific_icon_base}
\end{figure}

\todo[inline]{Insert a list of all icons along with a description of when to use them}
