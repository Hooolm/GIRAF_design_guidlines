%!TEX root = ../../main.tex

\chapter{Colors}
\index{Colors}
Throughout this chapter a single color in each entry will denote a solid color, while two colors in an entry will denote a gradient between the two colors.

\section{Text and Background}
\index{Colors!Text color}
\index{Colors!Background color}
These colors should be used throughout any application.

\begin{table}[!htbp]
	\begin{tabularx}{\textwidth}{l X r c}
		\collabel{1.1}
		& Regular text-color 
		& \texttt{\#000000} & \cellcolor[HTML]{000000}\phantom{--} \\ \hline
	\end{tabularx}
\end{table}

\begin{table}[!htbp]
	\begin{tabularx}{\textwidth}{l X r c}
		\collabel{1.2}
		& Text-color used to indicate placeholder or hint-texts 
		& \texttt{\#AAAAAA} & \cellcolor[HTML]{AAAAAA}\phantom{--} \\ \hline
	\end{tabularx}
\end{table}

\begin{table}[!htbp]
	\begin{tabularx}{\textwidth}{l X r c}
		\collabel{1.3}
		& Background-color for any applications window background 
		& \texttt{\#000000} & \cellcolor[HTML]{000000}\phantom{--} \\ \hline
	\end{tabularx}
\end{table}

\begin{table}[!htbp]
	\begin{tabularx}{\textwidth}{l X r c}
		\collabel{1.4}
		& Background-color for any activity 
		& \texttt{\#E9E9E9} & \cellcolor[HTML]{E9E9E9}\phantom{--} \\ \hline
	\end{tabularx}
\end{table}

\begin{table}[!htbp]
	\begin{tabularx}{\textwidth}{l X r c}
		\collabel{1.5}
		& Showcase help text-color 
		& \texttt{\#FFFFFF} & \cellcolor[HTML]{FFFFFF}\phantom{--} \\ \hline
	\end{tabularx}
\end{table}

\FloatBarrier

\section{Buttons}
\index{Colors!Button colors}
All buttons in the \giraf software suite should use these colors for buttons. Please note that all gradients defined below are from top to bottom. Also note that the colors for the disabled button must be slightly transparent ($65\%$).

\begin{table}[!htbp]
	\begin{tabularx}{\textwidth}{l X r c r c}
		\collabel{2.1}
		& Regular button background 
		& \texttt{\#FFCD59} & \cellcolor[HTML]{FFCD59}\phantom{--}
		& \texttt{\#FF9D00} & \cellcolor[HTML]{FF9D00}\phantom{--} \\ \hline
	\end{tabularx}
\end{table}

\begin{table}[!htbp]
	\begin{tabularx}{\textwidth}{l X r c r c}
		\collabel{2.2}
		& Regular button stroke/border 
		& ~ & ~
		& \texttt{\#8A6E00} & \cellcolor[HTML]{8A6E00}\phantom{--} \\ \hline
	\end{tabularx}
\end{table}

\begin{table}[!htbp]
	\begin{tabularx}{\textwidth}{l X r c r c}
		\collabel{2.3}
		& Pressed button background 
		& \texttt{\#D4AD2F} & \cellcolor[HTML]{D4AD2F}\phantom{--}
		& \texttt{\#FF9D00} & \cellcolor[HTML]{FF9D00}\phantom{--} \\ \hline
	\end{tabularx}
\end{table}

\begin{table}[!htbp]
	\begin{tabularx}{\textwidth}{l X r c r c}
		\collabel{2.4}
		& Pressed button stroke/border 
		& ~ & ~
		& \texttt{\#493700} & \cellcolor[HTML]{493700}\phantom{--} \\ \hline
	\end{tabularx}
\end{table}

\begin{table}[!htbp]
	\begin{tabularx}{\textwidth}{l X r c r c}
		\collabel{2.5}
		& Focused button background 
		& \texttt{\#FF9D00} & \cellcolor[HTML]{FF9D00}\phantom{--}
		& \texttt{\#FF5900} & \cellcolor[HTML]{FF5900}\phantom{--} \\ \hline
	\end{tabularx}
\end{table}

\begin{table}[!htbp]
	\begin{tabularx}{\textwidth}{l X r c r c}
		\collabel{2.6}
		& Focused button stroke/border 
		& ~ & ~
		& \texttt{\#8A6E00} & \cellcolor[HTML]{8A6E00}\phantom{--} \\ \hline
	\end{tabularx}
\end{table}

\begin{table}[!htbp]
	\begin{tabularx}{\textwidth}{l X r c r c}
		\collabel{2.7}
		& Focused button background 
		& \texttt{\#FAD355} & \cellcolor[HTML]{FAD355}\phantom{--}
		& \texttt{\#FEBE40} & \cellcolor[HTML]{FEBE40}\phantom{--} \\ \hline
	\end{tabularx}
\end{table}

\begin{table}[!htbp]
	\begin{tabularx}{\textwidth}{l X r c r c}
		\collabel{2.8}
		& Focused button stroke/border 
		& ~ & ~
		& \texttt{\#E4AE4E} & \cellcolor[HTML]{E4AE4E}\phantom{--} \\ \hline
	\end{tabularx}
\end{table}

\FloatBarrier

\section{Images}
\index{Image}
\index{Colors!Image}

\begin{table}[!htbp]
	\begin{tabularx}{\textwidth}{l X r c}
		\collabel{3.1}
		& Image background-color
		& \texttt{\#FFFFFF} & \cellcolor[HTML]{FFFFFF}\phantom{--} \\ \hline
	\end{tabularx}
\end{table}

\begin{table}[!htbp]
	\begin{tabularx}{\textwidth}{l X r c}
		\collabel{3.2}
		& Image border-color
		& \texttt{\#000000} & \cellcolor[HTML]{000000}\phantom{--} \\ \hline
	\end{tabularx}
\end{table}

\begin{table}[!htbp]
	\begin{tabularx}{\textwidth}{l X r c}
		\collabel{3.3}
		& Image marking-color
		& \texttt{\#FED76C} & \cellcolor[HTML]{FED76C}\phantom{--} \\ \hline
	\end{tabularx}
\end{table}

\FloatBarrier

\section{Bars}
\label{sec:color_bars}
\index{Bars}
\index{Colors!Bars}

All applications that using bars must use the following colors.

\begin{table}[!htbp]
	\begin{tabularx}{\textwidth}{l X r c r c}
		\collabel{4.1}
		& Background of any bar
		& \texttt{\#FDBB55} & \cellcolor[HTML]{FDBB55}\phantom{--}
		& \texttt{\#FED76C} & \cellcolor[HTML]{FED76C}\phantom{--} \\ \hline
	\end{tabularx}
\end{table}

\begin{table}[!htbp]
	\begin{tabularx}{\textwidth}{l X r c r c}
		\collabel{4.2}
		& Stroke/border of the bar 
		& ~ & ~
		& \texttt{\#E5BE53} & \cellcolor[HTML]{E5BE53}\phantom{--} \\ \hline
	\end{tabularx}
\end{table}

\begin{note}
	The gradient for the topbar is from top to bottom.
\end{note}

\begin{note}
	The gradient for the side is from left to right.
\end{note}

\begin{note}
	The gradient for the bottom is from bottom to top.
\end{note}

\FloatBarrier

\section{Week indicators}
\index{Week indicators}
\index{Colors!Week indicators}
These colors must be used whenever a certain weekday is referenced. Note that colors are primarily used to increase the usability for citizens.


\begin{table}[!htbp]
	\begin{tabularx}{\textwidth}{l X r c r c}
		\collabel{5.1}
		& Monday 
		& ~ & ~
		& \texttt{\#007700} & \cellcolor[HTML]{007700}\phantom{--} \\ \hline
	\end{tabularx}
\end{table}

\begin{table}[!htbp]
	\begin{tabularx}{\textwidth}{l X r c r c}
		\collabel{5.2}
		& Tuesday 
		& ~ & ~
		& \texttt{\#800080} & \cellcolor[HTML]{800080}\phantom{--} \\ \hline
	\end{tabularx}
\end{table}

\begin{table}[!htbp]
	\begin{tabularx}{\textwidth}{l X r c r c}
		\collabel{5.3}
		& Wednesday 
		& ~ & ~
		& \texttt{\#FF8500} & \cellcolor[HTML]{FF8500}\phantom{--} \\ \hline
	\end{tabularx}
\end{table}

\begin{table}[!htbp]
	\begin{tabularx}{\textwidth}{l X r c r c}
		\collabel{5.4}
		& Thursday 
		& ~ & ~
		& \texttt{\#0000FF} & \cellcolor[HTML]{0000FF}\phantom{--} \\ \hline
	\end{tabularx}
\end{table}

\begin{table}[!htbp]
	\begin{tabularx}{\textwidth}{l X r c r c}
		\collabel{5.5}
		& Friday 
		& ~ & ~
		& \texttt{\#FFDD00} & \cellcolor[HTML]{FFDD00}\phantom{--} \\ \hline
	\end{tabularx}
\end{table}

\begin{table}[!htbp]
	\begin{tabularx}{\textwidth}{l X r c r c}
		\collabel{5.6}
		& Saturday 
		& ~ & ~
		& \texttt{\#FF0000} & \cellcolor[HTML]{FF0000}\phantom{--} \\ \hline
	\end{tabularx}
\end{table}

\begin{table}[!htbp]
	\begin{tabularx}{\textwidth}{l X r c r c}
		\collabel{5.7}
		& Sunday 
		& ~ & ~
		& \texttt{\#FFFFFF} & \cellcolor[HTML]{FFFFFF}\phantom{--} \\ \hline
	\end{tabularx}
\end{table}

\FloatBarrier

\section{Page Indicator}
\index{Page indicator}
\index{Colors!Page indicator}
These colors must be used for indicating which page the user is currently on.

\begin{table}[!htbp]
	\begin{tabularx}{\textwidth}{l X r c r c}
		\collabel{6.1}
		& Active page
		& ~ & ~
		& \texttt{\#FF9D00} & \cellcolor[HTML]{FF9D00}\phantom{--} \\ \hline
	\end{tabularx}
\end{table}

\begin{table}[!htbp]
	\begin{tabularx}{\textwidth}{l X r c r c}
		\collabel{6.2}
		& Inactive page 
		& ~ & ~
		& \texttt{\#FFCD59} & \cellcolor[HTML]{FFCD59}\phantom{--} \\ \hline
	\end{tabularx}
\end{table}

\FloatBarrier