%!TEX root = ../../main.tex

\chapter{Application structure}
% TODO: Describe a generic application and how it should be structured. For instance the use of the topbar or sidemenu. Also talk about contextual menus.

\section{Top Bar}
\index{Action bar}
\index{Top bar}
All activities in every application, except the home-activity in the Launcher-application,  must have a top bar. This top bar should have a gradient a orange gradient (\colref{4.1}) as in the example in \figref{fig:top_bar_example}. This top bar should provide a short and simple title with enough information to allow the user to know where in the application he or she is. The height of the top bar must be $56dp$. Furthermore, this top bar must include at the two buttons described in \secref{sec:back_button} and \secref{sec:help_button}. A guide for customization of the action bar can be seen in \appref{app:customization_of_action_bar}.

\begin{figure}[!htbp]
        \centering
        \includegraphics[width=0.75\textwidth]{pictures/application_structure/topbar}
        \caption{Top bar example}
        \label{fig:top_bar_example}
\end{figure}

\begin{note}
    Such a top bar is easily achieved by letting all of your \androidinline{Activity} extend the \androidinline{GirafActivity} from the \gc library. Doing this will also implement the back button described in \secref{sec:back_button}.
\end{note}

\subsection{Back Button}
\index{Back button}
\index{Button}
\index{Button!Back button}
\label{sec:back_button}
The left-most button in the top bar must be a back button. This button must have exactly the same functionality as the back button on all Android devices. The icon of this button must be the back icon as described in \secref{sub:user_management}.

\subsection{Help Button}
\index{Help button}
\index{Button}
\index{Button!Help buttons}
\label{sec:help_button}
The right-most button in the left side of the top bar bar must be a help button. This button must provide the user with some useful help information regarding the current screen which the user is presented with. For instance if the user is assigning applications to users, the help might be a guide on how to do this correctly.

\FloatBarrier


\section{Side Bar}
\index{Side bar}

Side bars need not, but may be contextual. Side bars may be used to switch between content of applications - for instance if the application is split into two parts (side bar and content). A side bar should have a gradient a orange gradient (\colref{4.1}) and look like \figref{fig:side_bar_example}. Side bars may not use more than $20\%$ of the total screen estate. When using a side bar it must appear on the left side of the screen. The height of the side bar must be exactly the same as the container that it is inside. Most collections of items must have margins between them, the sidebar however have an exception regarding margin, meaning that there should be no margin between the markings of the items in the sidebar.

\begin{note}
    When using the \gc library one can access the layout resource \androidinline{R.layout.giraf_sidebar_layout} and inflate it. The parent of this is a \androidinline{RelativeLayout}.
\end{note}

\begin{figure}[!htbp]
        \centering
        \includegraphics[width=0.20\textwidth]{pictures/application_structure/sidebar}
        \caption{Side bar example}
        \label{fig:side_bar_example}
\end{figure}

\FloatBarrier


\section{Bottom bar}
\index{Bottom bar}
Bottom bars must be entirely contextual and depend on the current content displayed in the current activity. A bottom bar should have a gradient a orange gradient (\colref{4.1}) and look like \figref{fig:bottom_bar_example}.

\begin{note}
    When using the \gc library one can access the layout resource \androidinline{R.layout.giraf_bottom_layout} and inflate it. The parent of this is a \androidinline{RelativeLayout}.
\end{note}

\begin{figure}[!htbp]
    \centering
    \includegraphics[width=0.75\textwidth]{pictures/application_structure/bottombar}
    \caption{Bottom bar example}
    \label{fig:bottom_bar_example}
\end{figure}

\todo[inline]{\textbf{OBS:} The gradient is in the wrong direction. It should be light in the bottom darker in the top.}

\FloatBarrier


\section{Content}
\index{Layout content}
The main content of applications should focused in the center of the layout and any menu bars should be above, under, and to the sides of the main content. 

\begin{note}
We recommend using Android \androidinline{Fragment} instances to manage content of an \androidinline{Activity} if the main content of an Android \androidinline{Activity} needs to change between different content that needs to be controlled differently. 
\end{note}
\FloatBarrier

\section{Combinations of Bars}
\index{Bars}
You might want to combine usages of the bars mentioned in the previous sections. This section will describe the different possible combinations of bars.

\subsection{Combination \#1}
\index{Layout}
\index{Top bar}
\index{Bars!Top bar}
This is the most commonly used ``combination'' of bars (just the top bar actually). Since the top bar is required for all layouts this will be the base for the next combinations. \figref{fig:bar_combinations_1} shows an illustration of the layout using only the top bar. Please notice that the dark gray area is the top bar while the lighter gray is the actual content of the activity.

\begin{figure}[!htbp]
    \centering
    \includegraphics[width=0.35\textwidth]{pictures/application_structure/bar_combinations_1}
    \caption{Layout example with only the top bar}
    \label{fig:bar_combinations_1}
\end{figure}

\FloatBarrier

\subsection{Combination \#2}
\index{Layout}
\index{Side bar}
\index{Top bar}
\index{Bars!Top bar}
\index{Bars!Side bar}
If the content of the application needs to be switched out or replaced at some point, one may want to use a side bar. \figref{fig:bar_combinations_2} shows an illustration of such a layout using a top (darkest gray) bar and a side bar (dark gray). The content of the application is the light gray color.

\begin{figure}[!htbp]
    \centering
    \includegraphics[width=0.35\textwidth]{pictures/application_structure/bar_combinations_2}
    \caption{Layout example with both a side bar and a top bar}
    \label{fig:bar_combinations_2}
\end{figure}

\FloatBarrier

\subsection{Combination \#3}
\index{Layout}
\index{Side bar}
\index{Top bar}
\index{Bottom bar}
\index{Bars!Top bar}
\index{Bars!Side bar}
\index{Bars!Bottom bar}
If the content of the application needs to be switched out or replaced at some point, one may want to use a side bar. If the content that is replaced needs to be managed somehow a bottom bar can be used. \figref{fig:bar_combinations_3} shows an illustration of such a layout using a top (darker gray) bar, a side bar (dark gray), and a bottom bar (darkest gray). The content of the application is the light gray color.

\begin{figure}[!htbp]
    \centering
    \includegraphics[width=0.35\textwidth]{pictures/application_structure/bar_combinations_3}
    \caption{Layout example with both a side bar, a top bar, and a bottom bar}
    \label{fig:bar_combinations_3}
\end{figure}

\FloatBarrier

\section{Clickable Elements}
All elements that are clickable must have a safety-distance to other elements. This will ensure that the user does not accidentally press the wrong thing and ultimately does something wrong. This safety distance may be achieved using several different methods. Please refer to the following sections. \figref{fig:correct_element_spacing} shows an example of correct item spacing while \figref{fig:incorrect_element_spacing} shows an example of incorrect item spacing.
\\\\
Elements must have a safety distance to \ldots
\index{Margin}
\begin{itemize}
        \item Other clickable elements
        \item Borders of it's container
        \item Borders of the tablet
\end{itemize}

\begin{figure}[!htbp]
    \centering
    \begin{subfigure}[t]{0.4\textwidth}
        \centering
        \includegraphics[scale=0.1]{correct_element_spacing}
        \caption{Correct item spacing}
        \label{fig:correct_element_spacing}
    \end{subfigure}
    \hspace{5em} 
    \begin{subfigure}[t]{0.4\textwidth}
        \centering
        \includegraphics[scale=0.1]{incorrect_element_spacing}
        \caption{Incorrect item spacing}
        \label{fig:incorrect_element_spacing}
    \end{subfigure}
    
    \caption{Examples of correct and incorrect element spacing}
    \label{fig:element_spacing_examples}
\end{figure}

\subsection{Element Margin}
\index{Margin}
Elements may be spaced apart from each other using margin on the individual elements. The distance between the elements should be consistent throughout all activities of any given application. \figref{fig:element_margin_example} shows an example of the margin for a given element.. 

\begin{figure}[h]
        \centering
        \includegraphics[width=0.25\textwidth]{element_margin_example}
        \caption{Example of element margin}
        \label{fig:element_margin_example}
\end{figure}

\begin{note}
        If margin is used inside a container each element with margin will also be a certain distance from the borders of that specific container. If, for instance, an element has a margin of $10$, then this element would be a distance of $10$ from the borders of the container. This means that there \textit{might} not be need for any padding on the given container.
\end{note}

\subsubsection{Consistent Margin}
\index{Margin}
Elements of the same type appearing in the same context must have the same distance to other elements. However, if the elements appear in an order, for example a horizontal list, the first and last element may differ. For instance, the first element may have a smaller left-margin and the last element may have a smaller right-margin. \figref{fig:element_margin_consistency} shows an illustration of this example.

\begin{figure}[h]
        \centering
        \includegraphics[width=0.45\textwidth]{element_margin_consistency}
        \caption{Example of element margin}
        \label{fig:element_margin_consistency}
\end{figure}


\subsection{Container Padding}
\index{Padding}
Each container should provide some padding for its content. This padding should be somewhat identical to the spacing between elements inside the container. \figref{fig:container_padding_example} shows an example of a container with padding. 

\begin{figure}[h]
        \centering
        \includegraphics[width=0.65\textwidth]{container_padding_example}
        \caption{Example of a container with padding}
        \label{fig:container_padding_example}
\end{figure}

\begin{note}
        Whenever the content of the container can be scrolled through (for example a \texttt{GridView}) a property called \texttt{clipToPadding} must be set to \texttt{false} as seen in \figref{fig:clip_to_padding_false}. Please refer to the \href{http://developer.android.com/reference/android/view/ViewGroup.html#attr_android:clipToPadding}{Android documentation} for additional information.
\end{note}

\begin{figure}[!htbp]
    \centering
    \begin{subfigure}[t]{0.4\textwidth}
        \centering
        \includegraphics[scale=0.7]{clip_to_padding_false}
        \caption{\androidinline{clipToPadding} set to false}
        \label{fig:correct_element_spacing}
    \end{subfigure}
    \hspace{5em} 
    \begin{subfigure}[t]{0.4\textwidth}
        \centering
        \includegraphics[scale=0.7]{clip_to_padding_true}
        \caption{\androidinline{clipToPadding} set to true}
        \label{fig:incorrect_element_spacing}
    \end{subfigure}
    
    \caption{Illustration of the property \androidinline{clipToPadding}}
    \label{fig:clip_to_padding}
\end{figure}

\section{Item familiarity}
\label{sec:item_familiarity}

It is important that the users are familiar with the items they are manipulating. One should make it easy for the user to see that the items they are navigating, creating and editing are the same. For instance a category is displayed by an icon and a title below it. In the cases of creating and editing a category the layout and graphical presentation should be similar.

\todo[inline]{Insert image of the three phases of a category: create, display, update}

