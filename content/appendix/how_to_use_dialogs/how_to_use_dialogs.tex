%!TEX root = ../../../main.tex
\chapter{How to use dialogs}
\label{app:how_to_use_dialogs}

When using the various dialogs implemented in the \gc, it is important that one uses the support library, this is done by including the support library in the gradle build file as seen in \lstref{lst:build_gradle}.

\lstinputlisting[
    style = gradle,
    caption = {build.gradle},
    label = {lst:build_gradle},
]{content/appendix/how_to_use_dialogs/code_snippets/build.gradle}
\noindent
The dialogs is an extension of the android class \androidinline{DialogFragment}, meaning that all dialogs is handled as fragments. This means that callbacks from the dialogs is done using interfaces, which the activity starting them should implement. Through out the description of these dialogs there will be talked about a method called \androidinline{onActionButtonClick} which is an event that is called whenever some gui element is clicked, eg. when a button is clicked. This method will in all of the following examples create an instance of the dialog and show it using the \androidinline{show} method.
\\\\
Also note that in all of the following code snippets there is declared a string tag (\androidinline{DIALOG_TAG}) this is a tag that the Android system needs to handle fragments. However in many of the code snippets there also exist an integer (\androidinline{DIALOG_ID}) which is used to call a method in the activity from the fragment.

\begin{note}
	It is important that whenever one creates a dialog one uses the \androidinline{newInstance} method on the specific dialog.
\end{note}

\begin{note}
    Remember to call the \androidinline{show} method on the dialog when the dialog should be displayed. Simply instantiating it is not enough.
\end{note}

\begin{note}
    It is important to check that it is the correct \androidinline{dialogIdentifier} that is used in the methods implemented from interfaces in order to determine which dialog was actually responded to. 
\end{note}

\section{Confirm dialog}
\label{sec:impl_confirm_dialog}

The confirm dialog is used to make the user confirm an action before it is executed. This section describes how one could implement an confirm dialog as described in \secref{sec:confirm_dialog}. An example of such an implementation can be seen in \lstref{lst:impl_confirm_dialog}.
\\\\
Using the \androidinline{GirafConfrimDialog.Confirmation} interface (\lstrefline{lst:impl_confirm_dialog}{line:confirm_dialog:interface}) allows the activity (\androidinline{ExampleActivity}) to implement the method \androidinline{confirmDialog}, as in \lstrefline{lst:impl_confirm_dialog}{line:confirm_dialog:confirmdialogstart}, that will be called from the dialog whenever a response has been made for it (when the user press the acceptance button).  

\lstinputlisting[
    style = java,
    caption = {Implementaion of confirm dialog},
    label = {lst:impl_confirm_dialog},
]{content/appendix/how_to_use_dialogs/code_snippets/confirm_dialog.java}

\noindent
The default behavior of buttons in the dialog is to dismiss it. The acceptance button will furthermore call the \androidinline{confirmDialog} with the identifier \androidinline{CONFIRM_DIALOG_ID}, as in \lstreflines{lst:impl_confirm_dialog}{line:confirm_dialog:confirmdialogstart}{line:confirm_dialog:confirmdialogend}.

\section{Notify dialog}
\label{sec:impl_notify_dialog}
The notify dialog is used to make the user aware of some event. This section described how one could implement an notify dialog as described in \secref{sec:notify_dialog}. An example of such an implementation can be seen in \lstref{lst:impl_notify_dialog}.
\\\\
Using the \androidinline{GirafNotify.Notification} interface (\lstrefline{lst:impl_confirm_dialog}{line:notify_dialog:interface}) allows the activity (\androidinline{ExampleActivity}) to implement the method \androidinline{noticeDialog}, as in \lstrefline{lst:impl_notify_dialog}{line:notify_dialog:noticedialogstart}, that will be called from the dialog whenever a response has been made for it (when the user press the button).  

\lstinputlisting[
    style = java,
    caption = {Implementaion of notify dialog},
    label = {lst:impl_notify_dialog},
]{content/appendix/how_to_use_dialogs/code_snippets/notify_dialog.java}

\noindent
The default behavior the button in the dialog is to dismiss it. The button will furthermore call the \androidinline{confirmDialog} with the identifier \androidinline{NOTIFY_DIALOG_ID}, as in \lstreflines{lst:impl_confirm_dialog}{line:notify_dialog:noticedialogstart}{line:notify_dialog:noticedialogend}.

\section{Profileselector dialog}
\label{sec:impl_profileselector_dialog}

The profile selector dialog is used to allow for user to select one ore more profiles. This section describes how one could implement an confirm dialog as described in \secref{sec:profileselector_dialog}. Examples of such an implementation can be seen in \lstref{lst:impl_single_profileselector_dialog} and \lstref{lst:impl_multi_profileselector_dialog}.
\\\\
This dialog is used in two use cases. One where the user wants to select exactly one profiles, and another use case where one wants to select arbitrary many profiles. Using the \androidinline{GirafProfileSelectorDialog.OnSingleProfileSelectedListener} interface allows for the activity (\androidinline{ExampleActivity}) to respond on a single selected profile as in \lstrefline{lst:impl_single_profileselector_dialog}{line:single_profile_selector:interface}. Where as the \androidinline{GirafProfileSelectorDialog.OnMultipleProfilesSelectedListener} allows for the activity to respond on multi selected profiles an in \lstrefline{lst:impl_multi_profileselector_dialog}{line:multi_profile_selector:interface}.
\begin{note}
    Observe that the main difference between the instantiating of the single and multi version of the profile selector dialog is the fourth argument called \androidinline{selectMultipleProfiles}. That determines which interface will be called. If this boolean is set to \androidinline{true} the activity should implement \androidinline{GirafProfileSelectorDialog.OnMultiProfileSelectedListener} otherwise the activity should implement \androidinline{GirafProfileSelectorDialog.OnSingleProfileSelectedListener}. Obviously if an activity have one dialog of each type both interfaces should be implemented.
\end{note}

\lstinputlisting[
    style = java,
    caption = {Implementaion of single profile selector dialog},
    label = {lst:impl_single_profileselector_dialog},
]{content/appendix/how_to_use_dialogs/code_snippets/single_profile_selector_dialog.java}

\noindent
When a profile is clicked in a single profile selector the \androidinline{onProfileSelected} method is called, with the profile that was selected and the identifier \androidinline{PROFILE_SELECT_DIALOG_ID} as in \lstreflines{lst:impl_single_profileselector_dialog}{line:single_profile_selector:onprofileselectedstart}{line:single_profile_selector:onprofileselectedend}.

\lstinputlisting[
    style = java,
    caption = {Implementaion of multi profile selector dialog},
    label = {lst:impl_multi_profileselector_dialog},
]{content/appendix/how_to_use_dialogs/code_snippets/multi_profile_selector_dialog.java}

\noindent
When a profile is clicked in a multi profile selector the \androidinline{onProfilesSelected} method is called, with a list of profiles and a boolean telling of they were marked or not alongside the identifier \androidinline{PROFILE_SELECT_DIALOG_ID} as in \lstreflines{lst:impl_multi_profileselector_dialog}{line:multi_profile_selector:onprofilesselectedstart}{line:multi_profile_selector:onprofilesselectedend}.

\section{Waiting dialog}
\label{sec:impl_waiting_dialog}

The waiting dialog is used to indicate to the user that some task is being executed and that he/she should wait for that task to end. This section describes how one could implement an waiting dialog as described in \secref{sec:waiting_dialog}.

\begin{note}
    Note that in this dialog the dialog is instantiated in the \androidinline{onCreate} method (see \lstrefline{lst:impl_waiting_dialog}{line:waiting_dialog:oncreate}), and is called showed implicitly by the \androidinline{onActionButtonClick} method which starts the async task (see \lstrefline{lst:impl_waiting_dialog}{line:waiting_dialog:executetask}.
\end{note}

\noindent
This dialog is often used when some thread syncing tasks is running (see \appref{app:threading_asynctask}). This dialog requires not interface to work properly. Good practice with this dialog is to show it before the long task is executed. This can be done using the \androidinline{onPreExecute} method (see \lstrefline{lst:impl_waiting_dialog}{line:waiting_dialog:showdialog}). One should dismiss the dialog after the task is done which can be handled in \androidinline{onPostExecute} method (see \lstrefline{lst:impl_waiting_dialog}{line:waiting_dialog:dismissdialog}).

\lstinputlisting[
    style = java,
    caption = {Implementaion of waiting dialog},
    label = {lst:impl_waiting_dialog},
]{content/appendix/how_to_use_dialogs/code_snippets/waiting_dialog.java}

\section{Custom buttons dialog}
\label{sec:impl_custom_buttons}

The custom buttons dialog is used when the developer wants to provide the user with more than two buttons in the dialog. This section described how one coul dimplement an custom buttons dialog as described in \secref{sec:custom_buttons_dialog}.
\\\\
Using the \androidinline{GirafCustomButtonsDialog.CustomButtons} (\lstrefline{lst:impl_custom_buttons}{line:custom_buttons_dialog:interface}) interface allows for the developer to add abitrary many buttons to the dialog.

\begin{note}
    There is no limit on buttons that can be added to the dialog, but one should be careful that the design is consistent. The width should not exceed the width of other dialogs.
\end{note}

\lstinputlisting[
    style = java,
    caption = {Implementaion of custom buttons dialog},
    label = {lst:impl_custom_buttons},
]{content/appendix/how_to_use_dialogs/code_snippets/custom_buttons_dialog.java}

\noindent
Before the dialog is created the \androidinline{fillButtonContainer} method is called, see \lstreflines{lst:impl_custom_buttons}{line:custom_buttons_dialog:fillbuttonstart}{line:custom_buttons_dialog:fillbuttonend}. The identifier \androidinline{CUSTOM_BUTTONS_DIALOG_ID} determines which dialog is being filled with buttons.

\section{Inflatable dialog}
\label{sec:impl_inflatable_dialog}

The inflatable dialog is used to create more custom dialogs and want some custom gui elements to be shown in the dialog. In this example we have created a custom dialog that contains an AnalogClock element and a button to dismiss the dialog.
\\\\
When creating an instance of this type of dialog the \androidinline{newInstance} method takes an layout resource as third argument. The argument \androidinline{R.layout.example} as seen in \lstrefline{lst:impl_inflatable_dialog}{line:inflatable_dialog:layoutres} comes from the layout seen in \lstref{lst:impl_inflatable_dialog_layout}.

\lstinputlisting[
    style = java,
    caption = {The implementation of the inflatable dialog},
    label = {lst:impl_inflatable_dialog},
]{content/appendix/how_to_use_dialogs/code_snippets/inflatable_dialog.java}

\noindent
Using the \androidinline{GirafInflatableDialog.OnCustomViewCreatedListener} interface (\lstrefline{lst:impl_inflatable_dialog}{line:inflatable_dialog:interface}) allows for the developer to access the custom created layout by the \androidinline{editCustomView} method. See \lstreflines{lst:impl_inflatable_dialog}{line:inflatable_dialog:accesscustomviewstart}{line:inflatable_dialog:accesscustomviewend} for an example of how one access the custom view. 

\begin{note}
    One cannot access the custom inflated view before the \androidinline{editCustomView} is called from the dialog it self.
\end{note}

\lstinputlisting[
    style = xml,
    caption = {The custom layout for an inflateable dialog},
    label = {lst:impl_inflatable_dialog_layout},
]{content/appendix/how_to_use_dialogs/code_snippets/example_layout.xml}
