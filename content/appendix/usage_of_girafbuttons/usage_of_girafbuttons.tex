%!TEX root = ../../../main.tex
\chapter{Usage of GirafButtons}
\label{app:usage_of_girafbuttons}
\index{Button}
\index{GirafButton}

This appendix explains how ones used the class \androidinline{GirafButton} for complying wit the rules described in \charef{cha:buttons}. Through out this explanation we will be creating a buttons looking somewhat like the one in \figref{fig:girafbutton_camera}.

\begin{figure}[h]
	\centering
	\includegraphics[width=0.2\textwidth]{girafbutton_camera}
	\caption{Visual representaion of a \androidinline{GirafButton}}
	\label{fig:girafbutton_camera}
\end{figure}
\noindent

\lstinputlisting[
    style = java,
    caption = {GirafButton xml},
    label = {lst:button_xml},
]{content/appendix/usage_of_girafbuttons/example_activity.xml}

A \androidinline{GirafButton} can be inserted into the GUI both using xml component but also using a java class. If you have layout as in \lstref{lst:button_xml}, if you want to access the button you should use the \androidinline{findViewById} as seen in \lstrefline{line:button_java:findviewbyid}{lst:button_activity}.

\begin{note}
	When using xml to implement a button note that you have to set either the \androidinline{app:text} property or the \androidinline{app:icon} property. Otherwise an \androidinline{IllegalArgumentException} will be thrown.
\end{note}

\lstinputlisting[
    style = java,
    caption = {GirafButton activity},
    label = {lst:button_activity},
]{content/appendix/usage_of_girafbuttons/ExampleActivity.java}
\noindent
If one wants to create various buttons dynamically in code one can use one the three constructors provided for the \androidinline{GirafButton} class. These three constructors takes either a \androidinline{String} setting the text on the button, a \androidinline{Drawable} setting the icon or both as seen in \lstreflines{line:button_java:constructorstart}{line:button_java:constructorend}{lst:button_activity}.