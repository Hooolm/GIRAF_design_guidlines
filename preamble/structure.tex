%%%%%%%%%%%%%%%%%%%%%%%%%%%%%%%%%%%%%%%%%
% The Legrand Orange Book
% Structural Definitions File
% Version 2.0 (9/2/15)
%
% Original author:
% Mathias Legrand (legrand.mathias@gmail.com) with modifications by:
% Vel (vel@latextemplates.com)
% 
% This file has been downloaded from:
% http://www.LaTeXTemplates.com
%
% License:
% CC BY-NC-SA 3.0 (http://creativecommons.org/licenses/by-nc-sa/3.0/)
%
%%%%%%%%%%%%%%%%%%%%%%%%%%%%%%%%%%%%%%%%%

%----------------------------------------------------------------------------------------
%	VARIOUS REQUIRED PACKAGES AND CONFIGURATIONS
%----------------------------------------------------------------------------------------

\usepackage[top=3cm,bottom=3cm,left=3cm,right=3cm,headsep=10pt,a4paper]{geometry} % Page margins

\usepackage[table,xcdraw,usenames, dvipsnames]{xcolor} % Required for specifying colors by name
\usepackage{colortbl}

\usepackage{longtable}

%----------------------------------------------------------------------------------------
%	INPUT COLORS
%----------------------------------------------------------------------------------------
%!TEX root =  ../main.tex

% Define the orange color used for highlighting throughout the book
\definecolor{ocre}{RGB}{255,157,0} 

% Colors for examples that are wrong or right
\definecolor{rightexample}{RGB}{31,181,0}
\definecolor{wrongexample}{RGB}{238,41,18}

% Define colors for code
\definecolor{code_blue}{RGB}{46,55,247}
\definecolor{code_red}{RGB}{147,86,135}
\definecolor{code_gray}{rgb}{0.4,0.4,0.4}
\definecolor{code_darkblue}{rgb}{0.0,0.0,0.6}
\definecolor{code_cyan}{rgb}{0.0,0.6,0.6}
\definecolor{code_darkgreen}{rgb}{0,0.5,0}




% Used for longtables
\usepackage{tabularx}
\usepackage{longtable}
\usepackage{ltxtable}

\usepackage{graphicx} % Required for including pictures
\graphicspath{{Pictures/}} % Specifies the directory where pictures are stored

\usepackage{placeins} % Required for floatbarrier

\usepackage{lipsum} % Inserts dummy text

\usepackage{tikz} % Required for drawing custom shapes

\usepackage[english]{babel} % English language/hyphenation

\usepackage{enumitem} % Customize lists
\setlist{nolistsep} % Reduce spacing between bullet points and numbered lists

\usepackage{booktabs} % Required for nicer horizontal rules in tables

\usepackage{todonotes} % Used in order to make todos

\usepackage{xspace} % Used to make space in commands

% Used for making multiple figures within one figure
\usepackage{wrapfig}
\usepackage{subcaption}

\usepackage{array}

%----------------------------------------------------------------------------------------
%	FONTS
%----------------------------------------------------------------------------------------

\usepackage{avant} % Use the Avantgarde font for headings
%\usepackage{times} % Use the Times font for headings
\usepackage{mathptmx} % Use the Adobe Times Roman as the default text font together with math symbols from the Sym­bol, Chancery and Com­puter Modern fonts

\usepackage{microtype} % Slightly tweak font spacing for aesthetics
\usepackage[utf8]{inputenc} % Required for including letters with accents
\usepackage[T1]{fontenc} % Use 8-bit encoding that has 256 glyphs

%----------------------------------------------------------------------------------------
%	BIBLIOGRAPHY AND INDEX
%----------------------------------------------------------------------------------------

\usepackage[style=alphabetic,citestyle=numeric,sorting=nyt,sortcites=true,autopunct=true,babel=hyphen,hyperref=true,abbreviate=false,backref=true,backend=biber]{biblatex}
\addbibresource{bibliography.bib} % BibTeX bibliography file
\defbibheading{bibempty}{}

\usepackage{calc} % For simpler calculation - used for spacing the index letter headings correctly
\usepackage{makeidx} % Required to make an index
\makeindex % Tells LaTeX to create the files required for indexing

%----------------------------------------------------------------------------------------
%	MAIN TABLE OF CONTENTS
%----------------------------------------------------------------------------------------

\usepackage{titletoc} % Required for manipulating the table of contents

\contentsmargin{0cm} % Removes the default margin

% Part text styling
\titlecontents{part}[0cm]
{\addvspace{20pt}\centering\large\bfseries}
{}
{}
{}

% Chapter text styling
\titlecontents{chapter}[1.25cm] % Indentation
{\addvspace{12pt}\large\sffamily\bfseries} % Spacing and font options for chapters
{\color{ocre!60}\contentslabel[\Large\thecontentslabel]{1.25cm}\color{ocre}} % Chapter number
{\color{ocre}}  
{\color{ocre!60}\normalsize\;\titlerule*[.5pc]{.}\;\thecontentspage} % Page number

% Section text styling
\titlecontents{section}[1.25cm] % Indentation
{\addvspace{3pt}\sffamily\bfseries} % Spacing and font options for sections
{\contentslabel[\thecontentslabel]{1.25cm}} % Section number
{}
{\hfill\color{black}\thecontentspage} % Page number
[]

% Subsection text styling
\titlecontents{subsection}[1.25cm] % Indentation
{\addvspace{1pt}\sffamily\small} % Spacing and font options for subsections
{\contentslabel[\thecontentslabel]{1.25cm}} % Subsection number
{}
{\ \titlerule*[.5pc]{.}\;\thecontentspage} % Page number
[]

% List of figures
\titlecontents{figure}[0em]
{\addvspace{-5pt}\sffamily}
{\thecontentslabel\hspace*{1em}}
{}
{\ \titlerule*[.5pc]{.}\;\thecontentspage}
[]

% List of tables
\titlecontents{table}[0em]
{\addvspace{-5pt}\sffamily}
{\thecontentslabel\hspace*{1em}}
{}
{\ \titlerule*[.5pc]{.}\;\thecontentspage}
[]

%----------------------------------------------------------------------------------------
%	MINI TABLE OF CONTENTS IN PART HEADS
%----------------------------------------------------------------------------------------

% Chapter text styling
\titlecontents{lchapter}[0em] % Indenting
{\addvspace{15pt}\large\sffamily\bfseries} % Spacing and font options for chapters
{\color{ocre}\contentslabel[\Large\thecontentslabel]{1.25cm}\color{ocre}} % Chapter number
{}  
{\color{ocre}\normalsize\sffamily\bfseries\;\titlerule*[.5pc]{.}\;\thecontentspage} % Page number

% Section text styling
\titlecontents{lsection}[0em] % Indenting
{\sffamily\small} % Spacing and font options for sections
{\contentslabel[\thecontentslabel]{1.25cm}} % Section number
{}
{}

% Subsection text styling
\titlecontents{lsubsection}[.5em] % Indentation
{\normalfont\footnotesize\sffamily} % Font settings
{}
{}
{}

%----------------------------------------------------------------------------------------
%	PAGE HEADERS
%----------------------------------------------------------------------------------------

\usepackage{fancyhdr} % Required for header and footer configuration

\pagestyle{fancy}
\renewcommand{\chaptermark}[1]{\markboth{\sffamily\normalsize\bfseries\chaptername\ \thechapter.\ #1}{}} % Chapter text font settings
\renewcommand{\sectionmark}[1]{\markright{\sffamily\normalsize\thesection\hspace{5pt}#1}{}} % Section text font settings
\fancyhf{} \fancyhead[LE,RO]{\sffamily\normalsize\thepage} % Font setting for the page number in the header
\fancyhead[LO]{\rightmark} % Print the nearest section name on the left side of odd pages
\fancyhead[RE]{\leftmark} % Print the current chapter name on the right side of even pages
\renewcommand{\headrulewidth}{0.5pt} % Width of the rule under the header
\addtolength{\headheight}{2.5pt} % Increase the spacing around the header slightly
\renewcommand{\footrulewidth}{0pt} % Removes the rule in the footer
\fancypagestyle{plain}{\fancyhead{}\renewcommand{\headrulewidth}{0pt}} % Style for when a plain pagestyle is specified

% Removes the header from odd empty pages at the end of chapters
\makeatletter
\renewcommand{\cleardoublepage}{
\clearpage\ifodd\c@page\else
\hbox{}
\vspace*{\fill}
\thispagestyle{empty}
\newpage
\fi}

%----------------------------------------------------------------------------------------
%	THEOREM STYLES
%----------------------------------------------------------------------------------------

\usepackage{amsmath,amsfonts,amssymb,amsthm} % For math equations, theorems, symbols, etc

\newcommand{\intoo}[2]{\mathopen{]}#1\,;#2\mathclose{[}}
\newcommand{\ud}{\mathop{\mathrm{{}d}}\mathopen{}}
\newcommand{\intff}[2]{\mathopen{[}#1\,;#2\mathclose{]}}
\newtheorem{notation}{Notation}[chapter]

% Boxed/framed environments
\newtheoremstyle{ocrenumbox}% % Theorem style name
{0pt}% Space above
{0pt}% Space below
{\normalfont}% % Body font
{}% Indent amount
{\small\bf\sffamily\color{ocre}}% % Theorem head font
{\;}% Punctuation after theorem head
{0.25em}% Space after theorem head
{\small\sffamily\color{ocre}\thmname{#1}\nobreakspace\thmnumber{\@ifnotempty{#1}{}\@upn{#2}}% Theorem text (e.g. Theorem 2.1)
\thmnote{\nobreakspace\the\thm@notefont\sffamily\bfseries\color{black}---\nobreakspace#3.}} % Optional theorem note
\renewcommand{\qedsymbol}{$\blacksquare$}% Optional qed square

\newtheoremstyle{blacknumex}% Theorem style name
{5pt}% Space above
{5pt}% Space below
{\normalfont}% Body font
{} % Indent amount
{\small\bf\sffamily}% Theorem head font
{\;}% Punctuation after theorem head
{0.25em}% Space after theorem head
{\small\sffamily{\tiny\ensuremath{\blacksquare}}\nobreakspace\thmname{#1}\nobreakspace\thmnumber{\@ifnotempty{#1}{}\@upn{#2}}% Theorem text (e.g. Theorem 2.1)
\thmnote{\nobreakspace\the\thm@notefont\sffamily\bfseries---\nobreakspace#3.}}% Optional theorem note

\newtheoremstyle{blacknumbox} % Theorem style name
{0pt}% Space above
{0pt}% Space below
{\normalfont}% Body font
{}% Indent amount
{\small\bf\sffamily}% Theorem head font
{\;}% Punctuation after theorem head
{0.25em}% Space after theorem head
{\small\sffamily\thmname{#1}\nobreakspace\thmnumber{\@ifnotempty{#1}{}\@upn{#2}}% Theorem text (e.g. Theorem 2.1)
\thmnote{\nobreakspace\the\thm@notefont\sffamily\bfseries---\nobreakspace#3.}}% Optional theorem note

% Non-boxed/non-framed environments
\newtheoremstyle{ocrenum}% % Theorem style name
{5pt}% Space above
{5pt}% Space below
{\normalfont}% % Body font
{}% Indent amount
{\small\bf\sffamily\color{ocre}}% % Theorem head font
{\;}% Punctuation after theorem head
{0.25em}% Space after theorem head
{\small\sffamily\color{ocre}\thmname{#1}\nobreakspace\thmnumber{\@ifnotempty{#1}{}\@upn{#2}}% Theorem text (e.g. Theorem 2.1)
\thmnote{\nobreakspace\the\thm@notefont\sffamily\bfseries\color{black}---\nobreakspace#3.}} % Optional theorem note
\renewcommand{\qedsymbol}{$\blacksquare$}% Optional qed square
\makeatother

% Defines the theorem text style for each type of theorem to one of the three styles above
\newcounter{dummy} 
\numberwithin{dummy}{section}
\theoremstyle{ocrenumbox}
\newtheorem{theoremeT}[dummy]{Theorem}
\newtheorem{problem}{Problem}[chapter]
\newtheorem{exerciseT}{Exercise}[chapter]
\theoremstyle{blacknumex}
\newtheorem{exampleT}{Example}[chapter]
\theoremstyle{blacknumbox}
\newtheorem{vocabulary}{Vocabulary}[chapter]
\newtheorem{definitionT}{Definition}[section]
\newtheorem{corollaryT}[dummy]{Corollary}
\newtheorem{noteT}[dummy]{Note}
\theoremstyle{ocrenum}
\newtheorem{proposition}[dummy]{Proposition}

%----------------------------------------------------------------------------------------
%	DEFINITION OF COLORED BOXES
%----------------------------------------------------------------------------------------

\RequirePackage[framemethod=default]{mdframed} % Required for creating the theorem, definition, exercise and corollary boxes

% Theorem box
\newmdenv[skipabove=7pt,
skipbelow=7pt,
backgroundcolor=black!5,
linecolor=ocre,
innerleftmargin=5pt,
innerrightmargin=5pt,
innertopmargin=5pt,
leftmargin=0cm,
rightmargin=0cm,
innerbottommargin=5pt]{tBox}

% Exercise box	  
\newmdenv[skipabove=7pt,
skipbelow=7pt,
rightline=false,
leftline=true,
topline=false,
bottomline=false,
backgroundcolor=ocre!10,
linecolor=ocre,
innerleftmargin=5pt,
innerrightmargin=5pt,
innertopmargin=5pt,
innerbottommargin=5pt,
leftmargin=0cm,
rightmargin=0cm,
linewidth=4pt]{eBox}

% Example box (Right)	  
\newmdenv[skipabove=7pt,
skipbelow=7pt,
rightline=false,
leftline=true,
topline=false,
bottomline=false,
backgroundcolor=rightexample!10,
linecolor=rightexample,
innerleftmargin=5pt,
innerrightmargin=5pt,
innertopmargin=5pt,
innerbottommargin=5pt,
leftmargin=0cm,
rightmargin=0cm,
linewidth=4pt]{erBox}

% Example box (Wrong)	  
\newmdenv[skipabove=7pt,
skipbelow=7pt,
rightline=false,
leftline=true,
topline=false,
bottomline=false,
backgroundcolor=wrongexample!10,
linecolor=wrongexample,
innerleftmargin=5pt,
innerrightmargin=5pt,
innertopmargin=5pt,
innerbottommargin=5pt,
leftmargin=0cm,
rightmargin=0cm,
linewidth=4pt]{ewBox}

% Definition box
\newmdenv[skipabove=7pt,
skipbelow=7pt,
rightline=false,
leftline=true,
topline=false,
bottomline=false,
linecolor=ocre,
innerleftmargin=5pt,
innerrightmargin=5pt,
innertopmargin=0pt,
leftmargin=0cm,
rightmargin=0cm,
linewidth=4pt,
innerbottommargin=0pt]{dBox}	

% Corollary box
\newmdenv[skipabove=7pt,
skipbelow=7pt,
rightline=false,
leftline=true,
topline=false,
bottomline=false,
linecolor=gray,
backgroundcolor=black!5,
innerleftmargin=5pt,
innerrightmargin=5pt,
innertopmargin=5pt,
leftmargin=0cm,
rightmargin=0cm,
linewidth=4pt,
innerbottommargin=5pt]{cBox}

% Note box
\newmdenv[skipabove=7pt,
skipbelow=7pt,
rightline=false,
leftline=true,
topline=false,
bottomline=false,
linecolor=gray,
backgroundcolor=black!5,
innerleftmargin=5pt,
innerrightmargin=5pt,
innertopmargin=5pt,
leftmargin=0cm,
rightmargin=0cm,
linewidth=4pt,
innerbottommargin=5pt]{nBox}

% Creates an environment for each type of theorem and assigns it a theorem text style from the "Theorem Styles" section above and a colored box from above
\newenvironment{theorem}{\begin{tBox}\begin{theoremeT}}{\end{theoremeT}\end{tBox}}
\newenvironment{exercise}{\begin{eBox}\begin{exerciseT}}{\hfill{\color{ocre}\tiny\ensuremath{\blacksquare}}\end{exerciseT}\end{eBox}}

\newenvironment{exampleR}{\begin{erBox}\begin{exampleT}}{\hfill{\color{ocre}\tiny}\end{exampleT}\end{erBox}}

\newenvironment{exampleW}{\begin{ewBox}\begin{exampleT}}{\hfill{\color{ocre}\tiny}\end{exampleT}\end{ewBox}}

\newenvironment{definition}{\begin{dBox}\begin{definitionT}}{\end{definitionT}\end{dBox}}	
\newenvironment{example}{\begin{exampleT}}{\hfill{\tiny\ensuremath{\blacksquare}}\end{exampleT}}		
\newenvironment{corollary}{\begin{cBox}\begin{corollaryT}}{\end{corollaryT}\end{cBox}}

\newenvironment{note}{\begin{nBox}\begin{noteT}}{\end{noteT}\end{nBox}}	

%----------------------------------------------------------------------------------------
%	REMARK ENVIRONMENT
%----------------------------------------------------------------------------------------

\newenvironment{remark}{\par\vspace{10pt}\small % Vertical white space above the remark and smaller font size
\begin{list}{}{
\leftmargin=35pt % Indentation on the left
\rightmargin=25pt}\item\ignorespaces % Indentation on the right
\makebox[-2.5pt]{\begin{tikzpicture}[overlay]
\node[draw=ocre!60,line width=1pt,circle,fill=ocre!25,font=\sffamily\bfseries,inner sep=2pt,outer sep=0pt] at (-15pt,0pt){\textcolor{ocre}{R}};\end{tikzpicture}} % Orange R in a circle
\advance\baselineskip -1pt}{\end{list}\vskip5pt} % Tighter line spacing and white space after remark

%----------------------------------------------------------------------------------------
%	SECTION NUMBERING IN THE MARGIN
%----------------------------------------------------------------------------------------

\makeatletter
\renewcommand{\@seccntformat}[1]{\llap{\textcolor{ocre}{\csname the#1\endcsname}\hspace{1em}}}                    
\renewcommand{\section}{\@startsection{section}{1}{\z@}
{-4ex \@plus -1ex \@minus -.4ex}
{1ex \@plus.2ex }
{\normalfont\large\sffamily\bfseries}}
\renewcommand{\subsection}{\@startsection {subsection}{2}{\z@}
{-3ex \@plus -0.1ex \@minus -.4ex}
{0.5ex \@plus.2ex }
{\normalfont\sffamily\bfseries}}
\renewcommand{\subsubsection}{\@startsection {subsubsection}{3}{\z@}
{-2ex \@plus -0.1ex \@minus -.2ex}
{.2ex \@plus.2ex }
{\normalfont\small\sffamily\bfseries}}                        
\renewcommand\paragraph{\@startsection{paragraph}{4}{\z@}
{-2ex \@plus-.2ex \@minus .2ex}
{.1ex}
{\normalfont\small\sffamily\bfseries}}

%----------------------------------------------------------------------------------------
%	PART HEADINGS
%----------------------------------------------------------------------------------------

% numbered part in the table of contents
\newcommand{\@mypartnumtocformat}[2]{%
\setlength\fboxsep{0pt}%
\noindent\colorbox{ocre!20}{\strut\parbox[c][.7cm]{\ecart}{\color{ocre!70}\Large\sffamily\bfseries\centering#1}}\hskip\esp\colorbox{ocre!40}{\strut\parbox[c][.7cm]{\linewidth-\ecart-\esp}{\Large\sffamily\centering#2}}}%
%%%%%%%%%%%%%%%%%%%%%%%%%%%%%%%%%%
% unnumbered part in the table of contents
\newcommand{\@myparttocformat}[1]{%
\setlength\fboxsep{0pt}%
\noindent\colorbox{ocre!40}{\strut\parbox[c][.7cm]{\linewidth}{\Large\sffamily\centering#1}}}%
%%%%%%%%%%%%%%%%%%%%%%%%%%%%%%%%%%
\newlength\esp
\setlength\esp{4pt}
\newlength\ecart
\setlength\ecart{1.2cm-\esp}
\newcommand{\thepartimage}{}%
\newcommand{\partimage}[1]{\renewcommand{\thepartimage}{#1}}%
\def\@part[#1]#2{%
\ifnum \c@secnumdepth >-2\relax%
\refstepcounter{part}%
\addcontentsline{toc}{part}{\texorpdfstring{\protect\@mypartnumtocformat{\thepart}{#1}}{\partname~\thepart\ ---\ #1}}
\else%
\addcontentsline{toc}{part}{\texorpdfstring{\protect\@myparttocformat{#1}}{#1}}%
\fi%
\startcontents%
\markboth{}{}%
{\thispagestyle{empty}%
\begin{tikzpicture}[remember picture,overlay]%
\node at (current page.north west){\begin{tikzpicture}[remember picture,overlay]%	
\fill[ocre!20](0cm,0cm) rectangle (\paperwidth,-\paperheight);
\node[anchor=north] at (4cm,-3.25cm){\color{ocre!40}\fontsize{220}{100}\sffamily\bfseries\@Roman\c@part}; 
\node[anchor=south east] at (\paperwidth-1cm,-\paperheight+1cm){\parbox[t][][t]{8.5cm}{
\printcontents{l}{0}{\setcounter{tocdepth}{1}}%
}};
\node[anchor=north east] at (\paperwidth-1.5cm,-3.25cm){\parbox[t][][t]{15cm}{\strut\raggedleft\color{white}\fontsize{30}{30}\sffamily\bfseries#2}};
\end{tikzpicture}};
\end{tikzpicture}}%
\@endpart}
\def\@spart#1{%
\startcontents%
\phantomsection
{\thispagestyle{empty}%
\begin{tikzpicture}[remember picture,overlay]%
\node at (current page.north west){\begin{tikzpicture}[remember picture,overlay]%	
\fill[ocre!20](0cm,0cm) rectangle (\paperwidth,-\paperheight);
\node[anchor=north east] at (\paperwidth-1.5cm,-3.25cm){\parbox[t][][t]{15cm}{\strut\raggedleft\color{white}\fontsize{30}{30}\sffamily\bfseries#1}};
\end{tikzpicture}};
\end{tikzpicture}}
\addcontentsline{toc}{part}{\texorpdfstring{%
\setlength\fboxsep{0pt}%
\noindent\protect\colorbox{ocre!40}{\strut\protect\parbox[c][.7cm]{\linewidth}{\Large\sffamily\protect\centering #1\quad\mbox{}}}}{#1}}%
\@endpart}
\def\@endpart{\vfil\newpage
\if@twoside
\if@openright
\null
\thispagestyle{empty}%
\newpage
\fi
\fi
\if@tempswa
\twocolumn
\fi}

%----------------------------------------------------------------------------------------
%	CHAPTER HEADINGS
%----------------------------------------------------------------------------------------

\newcommand{\thechapterimage}{}%
\newcommand{\chapterimage}[1]{\renewcommand{\thechapterimage}{#1}}%
\def\@makechapterhead#1{%
{\parindent \z@ \raggedright \normalfont
\ifnum \c@secnumdepth >\m@ne
\if@mainmatter
\begin{tikzpicture}[remember picture,overlay]
\node at (current page.north west)
{\begin{tikzpicture}[remember picture,overlay]
\node[anchor=north west,inner sep=0pt] at (0,0) {\includegraphics[width=\paperwidth]{\thechapterimage}};
\draw[anchor=west] (\Gm@lmargin,-5.5cm) node [line width=2pt,rounded corners=15pt,draw=ocre,fill=white,fill opacity=0.5,inner sep=15pt]{\strut\makebox[22cm]{}};
\draw[anchor=west] (\Gm@lmargin+.3cm,-5.5cm) node {\huge\sffamily\bfseries\color{black}\thechapter. #1\strut};
\end{tikzpicture}};
\end{tikzpicture}
\else
\begin{tikzpicture}[remember picture,overlay]
\node at (current page.north west)
{\begin{tikzpicture}[remember picture,overlay]
\node[anchor=north west,inner sep=0pt] at (0,0) {\includegraphics[width=\paperwidth]{\thechapterimage}};
\draw[anchor=west] (\Gm@lmargin,-5.5cm) node [line width=2pt,rounded corners=15pt,draw=ocre,fill=white,fill opacity=0.5,inner sep=15pt]{\strut\makebox[22cm]{}};
\draw[anchor=west] (\Gm@lmargin+.3cm,-5.5cm) node {\huge\sffamily\bfseries\color{black}#1\strut};
\end{tikzpicture}};
\end{tikzpicture}
\fi\fi\par\vspace*{150\p@}}}

%-------------------------------------------

\def\@makeschapterhead#1{%
\begin{tikzpicture}[remember picture,overlay]
\node at (current page.north west)
{\begin{tikzpicture}[remember picture,overlay]
\node[anchor=north west,inner sep=0pt] at (0,0) {\includegraphics[width=\paperwidth]{\thechapterimage}};
\draw[anchor=west] (\Gm@lmargin,-5.5cm) node [line width=2pt,rounded corners=15pt,draw=ocre,fill=white,fill opacity=0.5,inner sep=15pt]{\strut\makebox[22cm]{}};
\draw[anchor=west] (\Gm@lmargin+.3cm,-5.5cm) node {\huge\sffamily\bfseries\color{black}#1\strut};
\end{tikzpicture}};
\end{tikzpicture}
\par\vspace*{150\p@}}
\makeatother

%----------------------------------------------------------------------------------------
%	HYPERLINKS IN THE DOCUMENTS
%----------------------------------------------------------------------------------------

\usepackage{hyperref}
\hypersetup{hidelinks,backref=true,pagebackref=true,hyperindex=true,colorlinks=false,breaklinks=true,urlcolor= ocre,bookmarks=true,bookmarksopen=false,pdftitle={Title},pdfauthor={Author}}
\usepackage{bookmark}
\bookmarksetup{
open,
numbered,
addtohook={%
\ifnum\bookmarkget{level}=0 % chapter
\bookmarksetup{bold}%
\fi
\ifnum\bookmarkget{level}=-1 % part
\bookmarksetup{color=ocre,bold}%
\fi
}
}

%----------------------------------------------------------------------------------------
%	COMMANDS
%----------------------------------------------------------------------------------------

%!TEX root = ../../main.tex

% \translated
% Will indicate that a given string is translated from danish
% arg1    the original string
% arg2    the string that has been translated
\newcommand{\translated}[2] {
  \emph{``#1'' (translated: ``#2'')}
}

% \centerfig
% Will insert a figure with 75% textwidth in the horizontal center of the page.
% arg1 		graphics file to include.
% arg2 		the caption of the figure.
% arg3		the label of the figure.
\newcommand{\centerfig}[3] {
  \begin{figure}[htbp]
    \centering
    \includegraphics[width=0.75\textwidth]{#1}
    \caption{#2}
    \label{#3}
  \end{figure}
  \noindent
}

% \centerfigwithwidth
% Will insert a figure with a custom width in the horizontal center of the page.
% arg1 		graphics file to include.
% arg2		the caption of the figure.
% arg3		the label of the figure.
% arg4 		the width of the figure. For instance 0.5 for 50% of the page.
\newcommand{\centerfigwithwidth}[4] {
  \begin{figure}[!htbp]
    \centering
    \includegraphics[width=#4]{#1}
    \caption{#2}
    \label{#3}
  \end{figure}
  \noindent
}

% \wrapfig
% Will insert a figure wrapping with the text of the page.
% arg1		the graphics file to include.
% arg2 		the caption of the figure.
% arg3 		the label of the figure.
% arg4		what side the figure should wrap to. l for left, r for right.
\newcommand{\wrapfig}[4] {
  \begin{wrapfigure}{#4}{0.5\textwidth}
    \begin{center}
      \includegraphics[width=0.5\textwidth]{#1}
    \end{center}
    \caption{#2}
    \label{#3}
  \end{wrapfigure}
  \noindent
}

% \wrapfigwithwidth
% Will insert a figure wrapping with the text of the page.
% arg1		the graphics file to include.
% arg2 		the caption of the figure.
% arg3 		the label of the figure.
% arg4		what side the figure should wrap to. l for left, r for right.
% arg5 		the width of the figure. For instance 0.5 for 50% of the page. 
\newcommand{\wrapfigwithwidth}[5] {
  \begin{wrapfigure}{#4}{0.5\textwidth}
    \begin{center}
      \includegraphics[width=#5]{#1}
    \end{center}
    \caption{#2}
    \label{#3}
  \end{wrapfigure}
  \noindent
}
%!TEX root = ../../super_main.tex

% Format specification for all lexems 
\newcommand{\lexical}[1]{\texttt{\textbf{#1}}\xspace}

% Define how the justify-alignment should be
\newcommand*\justify{
	\fontdimen2\font = 0.4em	% interword space
	\fontdimen3\font = 0.2em	% interword stretch
	\fontdimen4\font = 0.1em	% interword shrink
	\fontdimen7\font = 0.1em	% extra space
	\hyphenchar\font = `\-		% allowing hyphenation
}

% Used for code
\newcommand{\mono}{\texttt}

% Inline TAInC Language style
\def\taincinline{\lstinline[style = taincinline]}

% Inline Java Language style
\def\javainline{\lstinline[style = javainline]}

% Hide entries in table of content (use \tocless)
\newcommand{\nocontentsline}[3]{}
\newcommand{\tocless}[2]{\bgroup\let\addcontentsline=\nocontentsline#1{#2}\egroup}
%!TEX root = ../../main.tex

% GIRAF
\newcommand{\giraf}[0]{\emph{GIRAF}\xspace}

% Launcher
\newcommand{\launcher}[0]{\emph{Launcher}\xspace}

% Category tool
\newcommand{\ct}[0]{\emph{Category tool}\xspace}

% C emphed
\renewcommand{\c}[0]{\emph{C}\xspace}

% C# emphed
\newcommand{\csharp}[0]{\emph{C\#}\xspace}

% Pictosearch
\newcommand{\ps}[0]{\emph{Pictosearch}\xspace}

% GIRAF Components
\newcommand{\gc}[0]{\giraf \emph{Components}\xspace}
%!TEX root = ../super_main.tex

% Makes normal page numbering
\newcommand{\normalpagenumbering}
{
  \label{lastRoman}
  \cleardoublepage
  \pagenumbering{arabic}
  \setcounter{page}{1}
  \afterpage{\fancyfoot[LE]{\thepage{}}}
  \afterpage{\fancyfoot[RO]{\thepage{}}}
}

% Make a new even side (So the next content appears on the right side).
\newcommand{\newevenside}{
  \ifthenelse{\isodd{\thepage}}{\newpage}{
    \newpage
    \phantom{placeholder} % Some phantom text. Required.
    \thispagestyle{empty} % Do not display header/footer text
    \newpage
  }
}

%!TEX root = ../../main.tex

% ===================== %
% == Page references == %
% ===================== %

% \onpageref
% References a page using the "Page"-prefix.
% arg1		the label to reference to.
\newcommand{\onpageref}[1]{Page \pageref{#1}\xspace}



% ======================= %
% == Figure references == %
% ======================= %

% \figref
% References a figure using the "Figure"-prefix.
% arg1 		the labelname of the figure to reference. 
\newcommand{\figref}[1]{Figure \ref{#1}\xspace}

% \figrefpage
% References a figure using the "Figure"-prefix. Will also display the page of the figure.
% arg1		the labelname of the figure to reference.
\newcommand{\figrefpage}[1]{\figref{#1} on \onpageref{#1}}



% ====================== %
% == Table references == %
% ====================== %

% \tabref
% References a table using the "Table"-prefix.
% arg1		the labelname of the tabel to reference.
\newcommand{\tabref}[1]{Table \ref{#1}\xspace}

% \tabrefpage
% References a table using the "Table"-prefix. Will also display the page of the table.
% arg1		the labelname of the tabel to reference.
\newcommand{\tabrefpage}[1]{\tabref{#1} on \onpageref{#1}}



% ======================== %
% == Chapter references == %
% ======================== %

% \charef
% References a chapter using the "Chapter"-prefix.
% arg1 		the labelname of the chapter to reference. 
\newcommand{\charef}[1]{Chapter \ref{#1}\xspace}

% \secrefpage
% References a chapter using the "Chapter"-prefix. Will also display the page of the chapter.
% arg1		the labelname of the chapter to reference.
\newcommand{\charefpage}[1]{\charef{#1} on \onpageref{#1}}



% ========================= %
% == Appendix references == %
% ========================= %

% \appref
% References a appendix using the "Appendix"-prefix.
% arg1 		the labelname of the appendix to reference. 
\newcommand{\appref}[1]{Appendix \ref{#1}\xspace}

% \secrefpage
% References a appendix using the "Appendix"-prefix. Will also display the page of the appendix.
% arg1		the labelname of the appendix to reference.
\newcommand{\apprefpage}[1]{\appref{#1} on \onpageref{#1}}



% ======================== %
% == Section references == %
% ======================== %

% \secref
% References a section using the "Section"-prefix.
% arg1 		the labelname of the section to reference. 
\newcommand{\secref}[1]{Section \ref{#1}\xspace}

% \secrefpage
% References a section using the "Section"-prefix. Will also display the page of the section.
% arg1		the labelname of the section to reference.
\newcommand{\secrefpage}[1]{\secref{#1} on \onpageref{#1}}



% =========================== %
% == Definition references == %
% =========================== %

% \defref
% References a definition using the "Definition"-prefix.
% arg1 		the labelname of the definition to reference. 
\newcommand{\defref}[1]{Definition \ref{#1}\xspace}

% \defrefpage
% References a definition using the "Definition"-prefix. Will also display the page of the definition.
% arg1		the labelname of the definition to reference.
\newcommand{\defrefpage}[1]{\defref{#1} on \onpageref{#1}}



% =========================== %
% == Lstlisting references == %
% =========================== %

% \lstref
% References a lstlisting using the "Code snippet"-prefix.
% arg1 		the labelname of the lstlisting to reference. 
\newcommand{\lstref}[1]{Code Snippet \ref{#1}\xspace}

% \lstrefpage
% References a lstlisting using the "Code snippet"-prefix. Will also display the page of the lstlisting.
% arg1		the labelname of the lstlisting to reference.
\newcommand{\lstrefpage}[1]{\lstref{#1} on \onpageref{#1}}

% \lineref
% References a specific line. Using the "Line x".
% arg1		the labelname of the line in the listing to reference.
\newcommand{\lineref}[1]{Line \ref{#1}\xspace}

% \lineref
% References a specific line. Using the "Line x".
% arg1		the labelname of the starting line in the listing to reference.
% arg2		the labelname of the starting line to reference.
\newcommand{\linesref}[2]{Lines \ref{#1}-\ref{#2}\xspace}


% \lstrefline
% References a specific line on a specific code snippet. Using the "Line x in Code snippet y".
% arg1		the labelname of the lstlisting to reference.
% arg2		the labelname of the line to reference.
\newcommand{\lstrefline}[2]{Line \ref{#2} in \lstref{#1}}

% lstreflines
% references specific lines on a specific code snippet.
% arg1		the labelname of the lstlisting to reference.
% arg2 		the labelname of the starting line to reference.
% arg3		the labelname of the ending line to reference.
\newcommand{\lstreflines}[3]{Lines \ref{#2}-\ref{#3} in \lstref{#1}}



% ======================== %
% == Grammar references == %
% ======================== %

% \graref
% References a grammar using the "Grammar"-prefix.
% arg1 		the labelname of the grammar to reference. 
\newcommand{\graref}[1]{Grammar \ref{#1}\xspace}

% \grarefpage
% References a grammar using the "Grammar"-prefix. Will also display the page of the grammar.
% arg1		the labelname of the lstlisting to reference.
\newcommand{\grarefpage}[1]{\graref{#1} on \onpageref{#1}}


% ========================= %
% == Equation references == %
% ========================= %

% \graref
% References a equation using the "Equation"-prefix.
% arg1 		the labelname of the equation to reference. 
\newcommand{\equref}[1]{Equation \ref{#1}\xspace}

% \grarefpage
% References a equation using the "Equation"-prefix. Will also display the page of the equation.
% arg1		the labelname of the lstlisting to reference.
\newcommand{\equrefpage}[1]{\equref{#1} on \onpageref{#1}}

% \graref
% References a equation using the "Equation"-prefix.
% arg1      the labelname of the equation to reference. 
\newcommand{\forref}[1]{Formula \ref{#1}\xspace}

% =========================== %
% == Color references == %
% =========================== %

% \collabel
% Creates a label for a color and writes the id out as well
% arg1 		the id of the color to reference. (Do NOT use "col:" just use the number)
\newcommand{\collabel}[1]{#1 \label{col:#1}}

% \colref
% References a color using the "Color"-prefix.
% arg1 		the id of the color to reference. (Do NOT use "col:" just use the number)
\newcommand{\colref}[1]{Color \hyperref[col:#1]{#1}\xspace}

% \colrefpage
% References a color using the "Color"-prefix. Will also display the page of the color.
% arg1 		the id of the color to reference. (Do NOT use "col:" just use the number)
\newcommand{\colrefpage}[1]{\colref{#1} on \onpageref{col:#1}}
%!TEX root = ../../main.tex

% Packages used
\usepackage{amssymb}
\usepackage{pifont}

% ================================================ %

% Checkmark  
\renewcommand{\checkmark}{\ding{51}} 	
\newcommand{\bcheckmark}{\ding{52}}		 

% Cross
\newcommand{\cross}{\ding{53}}			
\newcommand{\bcros}{\ding{54}}			

% Crossmark
\newcommand{\crossmark}{\ding{55}}		
\newcommand{\bcrosmark}{\ding{56}}		
%!TEX root = ../..main.tex

% \cm{}
% Allows centering while defining a width on a table column
% arg1 		the width of the cell
\newcolumntype{x}[1]{>{\centering\arraybackslash\hspace{0pt}}p{#1}}

\usepackage{diagbox}

