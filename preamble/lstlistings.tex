%!TEX root = ../../super_main.tex

% Packages used
\usepackage{listings}

% ================================================ %

\captionsetup[lstlisting]{
    format = listing
}

% Default style for all lstlistings
\lstset 
{
    backgroundcolor = \color{white},
    keywordstyle = \color{blue},
    commentstyle = \color{gray!75}\textit,
    stringstyle = \color{green},
    basicstyle = \scriptsize\ttfamily,
    numberstyle = \tiny,
    numbers = left,
    breaklines = true,
    breakatwhitespace=true,
    showstringspaces = false,
    tabsize = 3,
    captionpos = t,
    extendedchars = true,
    escapeinside = {//*}{\^^M}, % Use latex inside lstlistings. For instance for refferences.
    frame = tblr,
    backgroundcolor = \color{gray!5},
    xleftmargin = 3.5pt,
}

% Write "Code snippet" instead of "listing".
\renewcommand{\lstlistingname}{Code Snippet}

% General style for the whole lstlisting
\DeclareCaptionFont{white}{\color{white}}
\DeclareCaptionFormat{listing}{\colorbox{gray}{\parbox{0.9934\textwidth}{#1#2#3}}}
\captionsetup[lstlisting]{format = listing, labelfont = white, textfont = white}

% Import the Java language
\lstloadlanguages{Java}

% Custom lststyle named java
\lstdefinestyle{java}
{
    breaklines = true,
    language = Java,
    columns = fullflexible,
    stringstyle=\color{eclipse_blue},
    morekeywords=[1]{class, return}, 
    keywordstyle=[1]\color{eclipse_red}
}

% Custom lstinline for Java named javainline
\newcommand{\androidinline}[1]{\lstinline[style = java, basicstyle = \ttfamily\normalsize]{#1}}
